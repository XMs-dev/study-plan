\section{Предисловие}\label{kicad:introduction:pre}
Планируемый курс должен научить слушателей разводить печатные платы в KiCAD. В качестве устройства для примера предлагаю использовать USB-программатор для AVR микроконтроллеров. В идеале, в дальнейшем хотелось бы добавить информации о том, как проектируются радиоэлектронные устройства (QUCS, ngSPICE), а также как программировать микроконтроллеры.

KiCAD был выбран тем, что обладает доработанным интерфейсом и привычен пользователям, а главное он обладает русскоязычной документацией и русским комьюнити.

Программатор http://easyelectronics.ru/skorostnoj-avr-usb-pro\-gram\-ma\-tor-na-ft232rl-bez-vspomogatelnogo-kontrollera.html

Курс будет состоять из двух циклов, базовая часть и дополнительные сведения. Материалы и РЭ слушатели покупают на свои собственные деньги. 

\section{Подготовительная часть} 
Приступим к изучению <<разводки>> печатных плат. Что это такое --- <<разводка>>? Своими словами, это будет как создание реального устройства --- печатной платы, из принципиальной схемы радиоэлектронного устройства. 
На современных предприятиях эта профессия выделена отдельно, так как количество компонентов возросло, а их габариты уменьшились.
В проектировании печатных плат есть свои особенности: так, толщины дорожек зависят от максимального тока проходящего через них, зазоры между дорожками зависят от максимальной разности напряжений и проводимости диэлектрика. В высокочастотных схемах на задержку сигнала также влияет не только длина дорожек, но и их форма.

Т.\,к. наши курсы посвящены Свободному программному обеспечению, то и раcсказывать я буду про свободные программы.

Выбирать тут придётся из двух программ: это GNU EDA и KiCAD. Мы выбрали последнее, так как KiCAD  обладает документацией на русском языке и русскоязычным сообществом; из достоинств оппонента хочется упомянуть интегрированную среду для анализа работы схемы, основанную на ngSPICE. 

Целью первого цикла наших занятий будет создание простейшего USB-программатора для AVR микроконтроллеров. И хотя с практической стороны сейчас большую популярность набирают ARM микроконтроллеры (из-за цены и мощностей), работа с ними более специфична и зависит от производителя.

\subsection{Установка KiCAD} 
Начнём с установки необходимых компонентов. Есть множество вариантов: простейший --- это поставить из репозитория вашего дистрибутива; проблемы сборки под ОС типа Windows и Mac OS~X  мы не рассматриваем, хотя они несомненно существуют. Недостаток версии из репозитория заключается в том, что она может оказаться устаревшей и в ней может не быть некоторых компонентов.
Так, в последних версиях тестовой ветки добавили новый формат для хранения плат, возможность работать с точностью сетки до нанометров, а также возможность писать сценарии для kicad на python; также отключены настройки, создающие рамку по ГОСТ, и дополнительные поля спецификации.
Поэтому опишем процесс сборки KiCAD на вашей машине. Нам понадобятся:
\begin{enumerate}
 \item утилита для работы с распределённой системой контроля версий --- \emph{bazaar};
 \item макропакет, включающий утилиты для сборки --- \emph{build-essential};
 \item программа для создания конфигурационных файлов, а также файла автоматической сборки --- \emph{cmake};
 \item кросплатформенные графические библиотеки --- \emph{wxwidget} версии не ниже 2.8.11;
 \item набор библиотек для C++ --- \emph{boost};
 \item пакет \emph{pngcrush};
\end{enumerate}

\subsubsection{Инструкция по сборке}
Для начала нам необходимо выкачать исходники:\\
\texttt{\$ bzr branch lp:kicad kicad.bzr}\\
\texttt{\$ cd kicad.bzr/}\\
\texttt{\$ mkdir build}\\
\texttt{\$ cd build/}\\

Дело в том, что разработчики редко обновляют файл \texttt{INSTALL.txt}, поэтому во всех дополнительных опциях компиляции приходится разбираться вручную. Для этого введем:\\
\texttt{\$ cmake -i ..}\\
Здесь появится список опций компиляции и то, как они установлены:
\begin{itemize}
 \item \texttt{Variable Name: KICAD\_GOST}

       Эта опция отвечает за создание рамок чертежей по ГОСТ, а также за создание дополнительных полей для компонентов принципиальной схемы --- это нужно, чтобы автоматически сгенерировать спецификацию деталей;
 \item \texttt{Variable Name: KICAD\_SCRIPTING}\\
       \texttt{Variable Name: KICAD\_SCRIPTING\_MODULES}\\
       \texttt{Variable Name: KICAD\_SCRIPTING\_WXPYTHON}

       Все эти опции используются для подключения движка, позволяющего писать скрипты и инструменты, встраивая их в KiCAD; в том числе и инструменты с GUI;
 \item \texttt{Variable Name: KICAD\_TESTING\_VERSION}

       Обязательный флаг, говорит о том, что мы собираем тестовую версию kicad;
 \item \texttt{Variable Name: USE\_PCBNEW\_NANOMETRES}\\
       \texttt{Variable Name: USE\_PCBNEW\_SEXPR\_FILE\_FORMAT}

       Последние изменения, которые и хотелось бы использовать. Нанометры позволили связать имперскую дюймовую и европейскую миллиметровую сетки. Второй флаг позволяет работать с новым форматом хранения плат; этот формат создаётся с рассчётом совместной сетевой работы;
 \item \texttt{Variable Name: wxUSE\_UNICODE}

       Подключение юникода, неанглоязычным пользователем никогда не мешало.
\end{itemize}

Далее необходимо запустить компиляцию, для чего используется команда \texttt{make}. На многоядерных системах с числом ядер Х можно запустить команду с опцией \texttt{-j} и числом X+1 в качестве агрумента опции:\\
\texttt{\$ make -j X+1}


