\section{Предисловие}
Планируемый курс должен научить слушателей разводить печатные платы в KiCAD. В качестве устройства для примере предлагаю использовать USB-программатор для AVR микроконтроллеров. В идеале в дальнейшем хотелось бы добавить информациии о том как проектируются радиоэлектронные устройства (QUCS, ngSPICE), а также как программировать микроконтроллеры.

KiCAD был выбран тем,что обладает доработанным интерфейсом и привычен пользователям, а главное он обладает русскоязычной документацией и русским комьюнити.

Программатор http://easyelectronics.ru/skorostnoj-avr-usb-programmator-na-ft232rl-bez-vspomogatelnogo-kontrollera.html

Курс будет состоять из двух циклов, базовая часть и дополнительные сведения. Материалы и РЭ слушатели покупают на свои собственные деньги. 
