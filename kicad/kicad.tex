<<<<<<< HEAD
\part{KiCAD}

Планируемый курс, должен научить слушателей разводить печатные платы в
KiCAD. В качестве устройства для примере предлагаю использовать
USB-программатор для AVR микроконтроллеров. В идеале в дальнейшем
хотелось бы добавить информациии о том как проектируются
радиоэлектронные устройства(QUCS. ngSPICE) , а также как
программировать микроконтроллеры.

Курс будет состоять из двух циклов, базовая часть и дополнительные сведенья. 
Материалы и РЭ слушатели покупают на свои собственные деньги. 

\textbf{Оценка времени по занятиям}
\begin{itemize}
\item 2 занятия базовая Eeshema (работа создание модуля)

\item 1 cvpcb -связь УГО и footprint-ов , библиотеки и поиск

\item 2 PCBnew (работа - разводка двухслойной платы, создание
  footprnt-a)

\item 1 Генерация Gerbview и травление ЛУТ-ом (а также небольшой
  ликбез по пайке и пайка)

\item 2 Schhist работа над проектами используя git(не обязательно, но
  помогает хранить большие проекты)

\item 2 Продвинутая работа в EEshema в том числе и многостраничные
  схемы

\item 2 PCBnew -продвинуто, классы проводников , автоматическая
  разводка , footprint wizard,
\end{itemize}

KiCAD был выбран тем,что обладает доработанным интерфейсом и привычен пользователям ,а главное он обладает русскоязычной документацией и русским комьюнити.

Программатор
http://easyelectronics.ru/skorostnoj-avr-usb-programmator-na-ft232rl-bez-vspomogatelnogo-kontrollera.html
 
=======
\part{KiCAD}\label{kicad}
>>>>>>> 1d1c2a792315e5dab671ff1017b469bd4cdd9194
