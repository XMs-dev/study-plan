\section{Обзор и концепции языка}\label{common-lisp:introduction:review}
1) Базовые концепции
 - Списочная нотация и минималистичный синтаксис
   - Следствие: тотальное уменьшение кол-ва специальных случаев (исключений или нюансов), о которых надо всегда помнить (работа в основном с семантикой)
 - Единообразное представление программ и данных
   - Следствие(1): достаточно простое метапрограммирование (создание программ, пишущих программы)
   - Следствие(2): естесственное ``сцепление'' механизмов языка друг с другов (без лишних нюансов)
 - Создание макросов с использованием целевого языка (языка, в который раскрываются макросы)
   - Пояснение: макросы создаются с использованием тех же языковых средств, что используются при обычном описании программ (работающих на стадии выполнения)
   - Следствие: возможность использовать ``многоэтажное метапрограммирование'' с умеренной сложностью

2) Средства функционального программирования
 - Ф-ии как first-class объекты (полноправные граждане) и определение анонимных ф-ий (лямбда-ф-ии)
   - Следствие: высокая декларативность программирования с использованием отображающих (mapXX) и других функционалов
 - Определение ф-ий во время выполнения
   - Следствие: наличие механизма ``замыканий'' (возможности захватывать контекст выполнения)
 - 

 
