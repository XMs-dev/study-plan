\section{Обзор и концепции языка}\label{common-lisp:introduction:review}
\subsection{Базовые концепции}\label{common-lisp:introduction:review:base}
\subsubsection{Списочная нотация и минималистичный синтаксис}
\paragraph{Следствие:} тотальное уменьшение кол-ва специальных случаев (исключений или нюансов), о которых надо всегда помнить (работа в основном с семантикой)
\subsubsection{Единообразное представление программ и данных}
\paragraph{Следствие 1:} достаточно простое метапрограммирование (создание программ, пишущих программы)
\paragraph{Следствие 2:} естественное <<сцепление>> механизмов языка друг с другом (без лишних нюансов)
\subsubsection{Создание макросов с использованием целевого языка}
\paragraph{Пояснение:} макросы создаются с использованием тех же языковых средств, что используются при обычном описании программ (работающих на стадии выполнения)
\paragraph{Следствие:} возможность использовать <<многоэтажное метапрограммирование>> с умеренной сложностью

\subsection{Средства функционального программирования}
\subsubsection{Функции как first-class объекты}
\paragraph{Следствие:} высокая декларативность программирования с использованием отображающих (mapXX) и других функционалов
\subsubsection{Определение функций во время выполнения}
\paragraph{Следствие:} наличие механизма <<замыканий>> (возможности захватывать контекст выполнения)
