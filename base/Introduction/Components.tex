\section{Из чего состоит компьютер}\label{base:introduction:components}
\subsection{Общий обзор}\label{base:introduction:components:review}
Рассмотрим типичный персональный компьютер. В общем случае он состоит из устройства вывода изображения (монитор), большой шумящей коробки (системный блок) и разных устройств поменьше.
\begin{figure}[h!]
 \centering
 \includegraphics[width=0.5\textwidth]{base/Introduction/Computer.png}
 \label{base:introduction:components:review:computerpic}
 \caption{Устройство персонального компьютера:}
 \footnotesize
 \begin{enumerate}
  \item монитор;
  \item материнская плата;
  \item процессор;
  \item порт ATA;
  \item оперативная память;
  \item карты расширений;
  \item блок питания;
  \item дисковод;
  \item жёсткий диск;
  \item клавиатура;
  \item компьютерная мышь;
 \end{enumerate}
\end{figure}

\subsection{Системный блок}\label{base:introduction:components:case}
Начнём наше знакомство с основного компонента настольного компьютера --- \emph{системного блока} (англ.~\emph{computer case} или \emph{computer chassis}). Сам по себе системный блок (по какой-то совершенно неясной причине иногда называемый <<процессором>>; наука пытается найти объяснение этому явлению) является корпусом, в котором располагаются все основные компоненты.
\begin{figure}[h!]
 \centering
 \includegraphics[width=0.5\textwidth]{base/Introduction/Case.png}
 \caption{Типичный системный блок}
 \label{base:introduction:components:case:typicalcasepic}
\end{figure}
\begin{figure}[h!]
 \centering
 \includegraphics[width=0.5\textwidth]{base/Introduction/Case_wikipedia.jpg}
 \caption{Корпуса серверов Википедии}
 \label{base:introduction:components:case:wikipediacasepic}
\end{figure}
Внутреннее пространство поделено на несколько секций: отсек под устройства формата 5,25 дюймов, отсек под устройства формата 3,5 дюйма, место крепления блока питания и место для крепления материнской платы.

Корпуса могут быть разных форм и размеров. Относительно пропорций есть два устоявшихся форм-фактора: горизонтальные и вертикальные. Среди горизонтальных форм-факторов можно отметить:
\begin{description}
 \item[Desktop] $533\times419\times152$\,мм
 \item[FootPrint] $406\times406\times152$\,мм
 \item[SlimLine] $406\times406\times101$\,мм
 \item[UltraSlimLine] $381\times352\times75$\,мм
\end{description}
Также к горизонтальным форм-факторам можно причислить стоечные корпуса.

Вертикальные форм-факторы встречаются наиболее часто в офисах, домах, компьютерных классах. Они объеденены общим названием \emph{Tower} (minitower, miditower, bigtower). Среди причин их распространённости можно отметить больший объём по сравнению с горизонтальными, за счёт чего внутри остаётся больше места для вентиляции или дополнительных компонентов.

Вне зависимости от форм-фактора, системный блок является своеобразной <<печкой>>. Для охлаждения устройств, расположенных в нём, используется система вентиляции или циркуляции охлаждённой жидкости. Вентиляцию обеспечивают специальные вентиляторы (\emph{кулеры}, англ. \emph{cooler}), расположенные как на наиболее тепловыделяющих уст\-ройствах, так и на самом корпусе.
Довольно часто кулеры ставятся парно и однонаправлено, таким образом получается, что один задумает воздух внутрь, а другой выдувает наружу. За счёт этого обеспечивается постоянный поток воздуха, который оказывает охлаждающее действие.
Также кулеры устанавливают на одну из боковых стенок вертикальных форм-факторов, и иногда на вертикальные. Т.\,к.~охлаждение --- необходимый для нормального функционирования компьютера процесс, к его осуществлению надо подходить очень внимательно.
В частности, представляется недопустимым помещение системных блоков в тестные ниши современных компьютерных столов, т.\,к.~ограниченность пространства этих ниш делает задачу охлаждения труднорешаемой, а установленные средства воздушного охлаждения --- малоэффективными.

\subsubsection{Центральный процессор}\label{base:introduction:components:cpu}
Теперь рассмотрим \emph{центральный процессор} --- главное управляющее устройство, исполняющее машинные инструкции.
Изначально термин центральное процессорное устройство описывал специализированный класс логических машин, предназначенных для выполнения сложных компьютерных программ.
Вследствие довольно точного соответствия этого назначения функциям существовавших в то время компьютерных процессоров, он естественным образом был перенесён на сами компьютеры.
Начало применения термина и его аббревиатуры по отношению к компьютерным системам было положено в 1960-е годы.
Устройство, архитектура и реализация процессоров с тех пор неоднократно менялись, однако их основные исполняемые функции остались теми же, что и прежде.

\begin{figure}[h!]
 \centering
 \includegraphics[width=0.33\textwidth]{base/Introduction/CPUup.jpg}
 \includegraphics[width=0.33\textwidth]{base/Introduction/CPUdown.jpg}
 \label{base:introduction:components:cpu:cpupic}
 \caption{Центральный процессор}
\end{figure}

Главными характеристиками ЦПУ являются: \emph{тактовая частота}, \emph{производительность}, \emph{энергопотребление}, \emph{архитектура} и \emph{нормы литографического производственного процесса}. Рассмотрим их поподробнее:

\textbf{Производительность} --- это количественная характеристика скорости выполнения определённых операций на компьютере.
Чаще всего вычислительная мощность измеряется во флопсах (количество операций с плавающей запятой в секунду), а также производными от неё.
На данный момент принято причислять к суперкомпьютерам системы с вычислительной мощностью более 10 Терафлопс ($10\times10^{12}$ или десять триллионов флопс; для сравнения среднестатистический современный настольный компьютер имеет производительность порядка 0,1 Терафлопс).

Существует несколько сложностей при определении вычислительной мощности компьютера.
Во-первых, следует иметь в виду, что производительность системы может сильно зависеть от типа выполняемой задачи.
В частности, отрицательно сказывается на вычислительной мощности необходимость частого обмена данных между составляющими компьютерной системы, а также частое обращение к памяти.
В связи с этим выделяют пиковую вычислительную мощность --- гипотетически максимально возможное количество операций над числами с плавающей запятой в секунду, которое способен произвести данный компьютер.

\textbf{Тактовая частота} --- как следует из названия, определяет количество тактов, выполняемых за секунду. Условно можно переопределить это понятие как <<количество инструкций в секунду>>.
Среди потребителей имеется распространённое заблуждение о том, что процессоры с более высокой тактовой частотой всегда имеют более высокую производительность, чем процессоры с более низкой тактовой частотой.
На самом деле это не совсем так, т.\,к.~за один такт на разных процессорах может выполняться разное количество инструкций.

\textbf{Норма литографического процесса} --- описание технического процесса при производстве процессора. Наиболее важной характеристикой является разрешающая способность оборудования --- то есть, как следствие, плотность нанесения элементов.

\textbf{Архитектура} процессора --- количественная составляющая компонентов микроархитектуры процессора компьютера (например, регистр флагов или регистры процессора), рассматриваемая IT-специалистами в аспекте прикладной деятельности.
С точки зрения программиста --- совместимость с определённым набором команд, их структуры и способа исполнения.
С точки зрения аппаратной составляющей вычислительной системы --- это некий набор свойств и качеств, присущий целому семейству процессоров.
Имеются различные классификации архитектур процессоров, как по организации (например, по количеству и скорости выполнения команд: RISC, CISC), так и по назначению (например, специализированные графические).

\subsubsection{Оперативная память}\label{base:introduction:components:ram}
\begin{figure}[h!]
 \centering
 \includegraphics[width=0.5\textwidth]{base/Introduction/Memory.jpg}
 \label{base:introduction:components:ram:rampic}
 \caption{Плашки памяти DDRAM}
\end{figure}
\emph{Оперативно запоминающее устройство} --- энергозависимая плата, обеспечивающая временное хранение данных. Также часто синонимом ОЗУ выступает словосочетание <<\emph{оперативная память}>> --- по сути, его основная функция.
В оперативной памяти хранятся данные приложений, операционная система и многое другое.
Именно отсюда центральный процессор берёт инструкции для выполнения, записанные в программе, и именно сюда записывает их результат.
Даже если надо вывести простую строчку текста на экран, эта строчка будет размещена в оперативной памяти, и лишь затем будет считана оттуда для вывода.
Как следствие, к ней идёт постоянное обращение и происходит постоянный обмен данными.
За счёт этого растут требования к скорости памяти, отчего она получила своё название, а вместе с этим и стоимость.
Оперативная память стоит очень дорого (250--1875~руб. за гигабайт у ОЗУ при 1,5--3,75~руб. за гигабайт у жёстких дисков), притом что её использует каждое приложение для хранения данных.

Основными параметрами оперативной памяти являются \emph{объём} и \emph{частота шины}.
В качестве объёма указывается количество данных, которое может быть помещено в память.
Частота шины же показывает, сколько операций может быть совершено за еденицу времени, таким образом плата с большей частотой окажется быстрее платы аналогичного объёма, но с реже обновляемой шиной.
Стоит отметить, что на памяти указывается \emph{максимальная} частота, а не фактическая.
Реальная скорость также зависит от частоты шины на материнской плате, и из двух будет использоваться самая медленная.

У других устройств могут быть свои аналоги оперативной памяти, призванные снизить нагрузку на оную и/или ускорить получение данных.
Например, в процессорах, особенно в центральном, имеется \emph{кеш-память}, обладающая огромной скоростью даже по сравнению с оперативной, куда записываются самые частоиспользуемые инструкции или другие данные, вероятность обращения к которым крайне высока.
Аналогичные по функциям кеши имеют и некоторые другие устройства.
Объём их, как правило, очень мал, что не позволяет использовать их для ускорения работы системы

\subsubsection{Материнская плата}\label{base:introduction:components:motherboard}
\begin{figure}[h!]
 \centering
 \includegraphics[width=0.9\textwidth]{base/Introduction/Motherboard_diagram.png}
 \caption{Схема устройства материнской платы}
 \label{base:introduction:components:motherboard:diagrampic}
\end{figure}
\begin{figure}
 \centering
 \includegraphics[width=\textwidth]{base/Introduction/Motherboard.png}
 \label{base:introduction:components:motherboard:motherboardpic}
\end{figure}
Следующей рассмотрим основную плату компьютера, соединяющую всё воедино.
\emph{Материнская плата} (англ. \emph{motherboard}, \emph{MB}) --- сложная многослойная печатная плата, на которой устанавливаются основные компоненты персонального компьютера.
Именно материнская плата объединяет и координирует работу таких различных по своей сути и функциональности комплектующих, как процессор, оперативная память, платы расширения и всевозможные накопители.

Современные материнские платы содержат, как минимум:
\begin{itemize}
 \item сокет для ЦПУ;
 \item слоты для оперативной памяти;
 \item \emph{чипсет} (англ. \emph{chipset}), являющийся основным средством взаимодействия между устойствами;
 \item энергонезависимую память, хранящую BIOS и прочее необходимое ПО;
 \item тактовый генератор, используемый для синхронизации различных компонентов;
 \item слоты расширения;
 \item разъёмы для подключения внешнего источника питания, а также разъёмы для обеспечения питанием некоторых устройств (например, кулер ЦПУ).
\end{itemize}

Также, почти все материнские платы включают логику и разъёмы для поддержки часто используемых устройств ввода, таких как PS/2 для подключения мыши и клавиатуры.
Иногда они могут содержать графические интерфейсы и простые графические процессоры.
Дополнительная периферия, такая как дисковые контроллеры, может быть представлена платами расширения.

Учитывая большое тепловыделение быстрых процессоров и прочих устройств, современные материнские платы имеют радиаторы и точки для крепления дополнительных вентиляторов.

Чипсет --- это набор системной логики, набор микросхем, обеспечивающих подключение ЦПУ к ОЗУ и контроллерам периферийных устройств.
Как правило, современные наборы системной логики строятся на базе двух схем: <<северного>> и <<южного мостов>>.

\emph{Северный мост} (англ. \emph{Northbridge}), \emph{MCH} (\emph{Memory controller hub}), \emph{системный контроллер} --- обеспечивает подключение ЦПУ к узлам, использующим высокопроизводительные шины: ОЗУ, графический контроллер.
Обычно к системному контроллеру подключается ОЗУ, в этом случае он содержит в себе контроллер памяти.
Таким образом, от типа применённого системного контроллера обычно зависит максимальный объём ОЗУ, а также пропускная способность шины памяти персонального компьютера.
Однако в настоящее время имеется тенденция встраивания контроллера ОЗУ непосредственно в центральный процессор, что упрощает функции системного контроллера и снижает тепловыделение.
В качестве шины для подключения графического контроллера на современных материнских платах используется PCI Express.
Ранее использовались общие шины (ISA, VLB, PCI) и шина AGP.

\emph{Южный мост} (англ. \emph{Southbridge}), \emph{ICH} (\emph{I/O controller hub}), \emph{периферийный контроллер} --- содержит контроллеры периферийных устройств (таких как жёсткого диска, Ethernet, аудио), контроллеры шин для подключения периферийных устройств (шины PCI, PCI Express и USB), а также контроллеры шин, к которым подключаются устройства, не требующие высокой пропускной способности.

\subsubsection{Видеокарта}\label{base:introduction:components:videocard}
\begin{figure}[h!]
 \centering
 \includegraphics[width=0.5\textwidth]{base/Introduction/Videocard.jpg}
 \caption{Видеокарта}
 \label{base:introduction:components:videocard:videocardpic}
\end{figure}
Перейдём к платам расширения и начнём с одной и самых важных --- \emph{видеокарты}.
Видеокарта (\emph{графический адаптер}) преобразует графический образ, хранящийся в памяти компьютера или самого адаптера, в форму, пригодную для дальнейшего вывода на экран монитора.
В настоящее время, однако, эта базовая функция утратила основное значение, и, в первую очередь, под графическим адаптером понимают устройство с \emph{графическим процессором} --- \emph{графический ускоритель}, который и занимается формированием самого графического образа.
Современные видеокарты не ограничиваются простым выводом изображения, они имеют встроенный графический процессор, который может производить дополнительную обработку, снимая эту задачу с центрального процессора компьютера.
В последнее время также имеет место тенденция использовать вычислительные возможности графического процессора для решения неграфических задач.

Видеоускоритель может быть также интегрирован в материнскую плату или (с недавних пор) центральный процессор, однако все материнские платы настольных компьютеров имеют слот расширения под отдельную плату.
Современные компьютеры низшего и среднего классов часто содержат графический чипсет, расположенный на северном мосту материнской платы.
Эти графические чипы обычно содержат небольшой объём памяти (или не содержат её вовсе) и резервируют для своих нужд часть оперативной памяти, уменьшая её доступный остальным компонентам объём и приводя к ограничениям производительности, так как и центральный и графический процессоры для доступа к памяти используют одну шину.

Интегрированные видеокарты подразделяются на следующие категории:
\begin{itemize}
 \item \emph{Графика с разделяемой памятью} (\emph{Shared graphics}, \emph{Shared Memory Architecture}).
  Видеопамять в виде специализированных ячеек как таковая отсутствует; вместо этого под нужды видеоадаптера динамически выделяется область основной оперативной памяти компьютера.
  Такой способ адресации памяти почти исключительно используют т.\,н.~интегрированные видеокарты (т.\,е.~выполненные не в виде отдельной микросхемы, а являющиеся частью одного большого чипа --- северного моста).
  Преимущества данного решения --- низкая цена и малое энергопотребление.
  Недостатки --- невысокая производительность в 3D-графике и меньшая пропускная способность памяти.
 \item \emph{Дискретная графика} (\emph{Dedicated graphics}).
  На системной плате или (реже) на отдельном модуле распаяны видеочип и один или больше модулей видеопамяти.
  Только дискретная графика обеспечивает наивысшую производительность в трёхмерной графике.
  Недостатки: более высокая цена (для высокопроизводительных процессоров --- очень высокая) и большее энергопотребление.
 \item \emph{Гибридная дискретная графика} (\emph{Hybrid graphics}).
  Как следует из названия --- комбинация вышеназванных способов, ставшая возможной с появлением шины PCI Express.
  Наличествует небольшой объём физически распаянной на плате видеопамяти, который может виртуально расширяться за счёт использования основной оперативной памяти.
  Компромиссное решение, с разной степенью успеха пытаеющееся нивелировать недостатки двух вышеназванных видов, но не устраняет их полностью.
\end{itemize}

Среди характеристик видеоадаптеров следует выделить следующие:
\begin{itemize}
 \item \emph{ширина шины памяти}, измеряется в битах --- количество информации, передаваемой за такт.
  Важный параметр в производительности карты;
 \item \emph{объём видеопамяти} --- объём собственной оперативной памяти видеокарты.
  Видеокарты, интегрированные в набор системной логики материнской платы или являющиеся частью ЦПУ, обычно не имеют собственной видеопамяти и используют для своих нужд часть оперативной памяти компьютера;
 \item \emph{частоты ядра} и \emph{памяти} --- чем больше, тем быстрее видеокарта будет обрабатывать информацию;
 \item \emph{текстурная} и \emph{пиксельная скорость заполнения} --- показывает количество выводимой информации в единицу времени.
\end{itemize}

\subsubsection{Звуковая карта}\label{base:introduction:components:soundcard}
\begin{figure}
 \centering
 \includegraphics[width=0.5\textwidth]{base/Introduction/Soundcard.jpg}
 \label{base:introduction:components:soundcard:soundcardpic}
 \caption{Звуковая карта}
\end{figure}
\emph{Звуковая карта} (англ. \emph{sound card}) --- дополнительное оборудование персонального компьютера, позволяющее воспроизводить и записывать звук.
На момент появления звуковые платы представляли собой отдельные карты расширения, устанавливаемые в соответствующий слот.
В современных материнских платах представлены в виде интегрированного в материнскую плату аппаратного кодека.

Современные встроенные звуковые карты обеспечивают качество, достаточное для большинства пользователей.  Даже самые низкобюджетные платы могут воспроизводить аудио CD-качества, поэтому перечислять характеристики представляется излишним.

Разъёмы на звуковых картах покрашены в определённые цвета в зависимости от их роли (\hyperref[base:introduction:components:soundcard:connectorstable]{таблица \ref*{base:introduction:components:soundcard:connectorstable}}).
\begin{table}
 \definecolor{lightblue}{rgb}{0.61, 0.76, 0.81}
 \definecolor{gold}{rgb}{1, 0.84, 0}
 \definecolor{silver}{gray}{0.68}
 \centering
 \begin{tabular}{|c|>{\columncolor[gray]{0.9}}l|l|}
  \hline
  \multicolumn{2}{|>{\columncolor[gray]{0.9}}c|}{\textbf{Цвет}} & \multicolumn{1}{>{\columncolor[gray]{0.9}}c|}{\textbf{Функция}} \\ 
  \hline 
  \cellcolor{pink} & \textbf{Розовый} & Микрофонный вход \\ 
  \hline 
  \cellcolor{lightblue} & \textbf{Голубой} & Линейный вход \\ 
  \hline 
  \cellcolor{lime} & \textbf{Лайм} & Стереовыход, передние колонки \\ 
  \hline 
  \cellcolor{silver} & \textbf{Серый} & Боковые колонки \\ 
  \hline 
  \cellcolor{black} & \textbf{Чёрный} & Задние колонки \\ 
  \hline 
  \cellcolor{orange} & \textbf{Оранжевый} & Центральный динамик/сабвуфер \\ 
  \hline 
  \cellcolor{gold} & \textbf{Золотой} & Гейм-порт \\ 
  \hline 
 \end{tabular}
 \caption{Цветовые коды разъёмов}
 \label{base:introduction:components:soundcard:connectorstable}
\end{table}
Аналогичные цвета можно встретить и на периферийной акустике, что упрощает процесс коммутации и позволяет избежать ошибок. Иногда рядом с разъёмами находятся соответствующие изображения, позволяющие интуитивно догадаться о назначении разъёма.

Обычно звуковые карты имеют разъёмы 3,5~мм. RTS (также называемый <<миниджек>>), однако может встречаться RCA (<<тюльпан>>) для цифровой передачи данных, профессиональные 6,35~мм RTS (<<джек>>) и другие.

\subsubsection{Сетевая карта}\label{base:introduction:components:nic}
\begin{figure}[h!]
 \centering
 \includegraphics[width=0.5\textwidth]{base/Introduction/NIC.jpg}
 \caption{Сетевая карта}
 \label{base:introduction:components:nic:nicpic}
\end{figure}
\emph{Сетевая плата} (англ. \emph{network interface controller}) --- периферийное устройство, позволяющее компьютеру взаимодействовать с другими устройствами сети.
В настоящее время, особенно в персональных компьютерах, сетевые платы довольно часто интегрированы в материнские платы для удобства и удешевления всего компьютера в целом.

Доступ к сети может осуществляться как с помощью проводов, так и без них.
Для проводных сетевых карт характерно наличие разъёма 8P8C, внешне похожего на телефонный RJ-45.
Беспроводные платы вместо этого имеют антену, зачастую съёмную. Также на картах часть имеется светодиод, отображающий сетевую активность.
Серверный сетевые карты таке имеют собственный процессор, снимающий нагрузку по обработке сетевого потока с центрального процессора.

Из значимых характеристик современных сетевых карт для домашнего пользования можно назвать следующие:
\begin{itemize}
 \item скорость передачи данных;
 \item интерфейс (шина подключения);
 \item поддержка различных стандартов, таких как 802.1p, 802.3x, и~т\,д..
\end{itemize}

Сетевые карты обеспечивают передачу данных по сети, через них же осуществляется подключение к интернету. Подробнее сети, их организация и управление будут рассмотрены в \hyperref[base:networking]{главе \ref*{base:networking}}.

\subsubsection{Блок питания}\label{base:introduction:components:psu}


\subsection{Монитор}\label{base:introduction:components:monitor}


\subsection{Периферия}\label{base:introduction:components:peripheral}


\subsection{Принтеры, сканеры и МФУ}\label{base:introduction:components:printers}

