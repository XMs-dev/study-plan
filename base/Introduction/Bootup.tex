\section{Загрузка}\label{base:introduction:bootup}
\subsection{Начальная загрузка}\label{base:introduction:bootup:BIOS}
Большинство компьютерных систем могут исполнять только команды, находящиеся в оперативной памяти компьютера, в то время как современные операционные системы в большинстве случаев хранятся на жёстких дисках, загрузочных CD-ROM, USB дисках или в локальной сети.

После включения компьютера в его оперативной памяти нет операционной системы. Само по себе, без операционной системы, аппаратное обеспечение компьютера не может выполнять сложные действия, такие как, например, загрузку программы в память.
Таким образом мы сталкиваемся с парадоксом, который кажется неразрешимым: для того, чтобы загрузить операционную систему в память, мы уже должны иметь операционную систему в памяти.

Решением данного парадокса является использование специальной маленькой компьютерной программы, называемой \emph{BIOS} (\emph{Basic Input/Output System}). Эта программа не обладает всей функциональностью операционной системы, но её достаточно для того, чтобы загрузить другую программу, которая будет загружать операционную систему.
Часто используется многоуровневая загрузка, в которой несколько небольших программ вызывают друг друга до тех пор, пока одна из них не загрузит операционную систему.

В современных компьютерах процесс начальной загрузки начинается с выполнения процессором команд, расположенных в постоянной памяти (на IBM PC --- команд BIOS), начиная с предопределённого адреса (процессор делает это после перезагрузки без какой бы то ни было помощи).
Данное программное обеспечение может обнаруживать устройства, подходящие для загрузки, и загружать со специального раздела выбранного устройства (чаще всего загрузочного сектора данного устройства) загрузчик ОС. Загрузка операционной системы будет рассмотрена \hyperref[base:os:structure:bootandinit]{далее, в главе \ref*{base:os}, раздел \ref*{base:os:structure:bootandinit}}.

\subsection{Процедура POST}\label{base:introduction:bootup:post}
Особенно надо отметить процедуру \emph{POST} (\emph{power-on self-test}, самопроверка при включении), которая является серией тестов аппаратного обеспечения и BIOS. Функции, аналогичные POST компьютера, характерны для многих современных электронных устройств --- от ПЛК до смартфонов.

Сокращённый тест включает:
\begin{enumerate}
 \item Проверку целостности программ BIOS в ПЗУ, используя контрольную сумму;
 \item Обнаружение и инициализацию основных контроллеров, системных шин и подключенных устройств (графического адаптера, контроллеров дисководов и т. п.), устройств и обеспечивающих их самоинициализацию, а также выполнение программ, входящих в BIOS;
 \item Определение размера оперативной памяти и тестирования первого сегмента (64 килобайт).
\end{enumerate}

Полный регламент работы POST:
\begin{enumerate}
 \item Проверка регистров процессора;
 \item Проверка контрольной суммы ПЗУ;
 \item Проверка системного таймера и порта звуковой сигнализации (для IBM PC — ИМС i8255 или аналог);
 \item Тест контроллера прямого доступа к памяти;
 \item Тест регенератора оперативной памяти;
 \item Тест нижней области ОЗУ для проецирования резидентных программ в BIOS;
 \item Загрузка резидентных программ;
 \item Тест стандартного графического адаптера (VGA);
 \item Тест оперативной памяти;
 \item Тест основных устройств ввода (НЕ манипуляторов);
 \item Тест CMOS;
 \item Тест основных портов LPT/COM;
 \item Тест накопителей на гибких магнитных дисках (НГМД);
 \item Тест накопителей на жёстких магнитных дисках (НЖМД);
 \item Самодиагностика функциональных подсистем BIOS;
 \item Передача управления загрузчику.
\end{enumerate}

Выбор между прохождением полного или сокращенного набора тестов при включении компьютера можно задать в программе настройки базовой системы ввода-вывода, BIOS setup.

В большинстве персональных компьютеров в случае успешного прохождения POST системный динамик издаёт один короткий звуковой сигнал, в случае сбоя --- различные последовательности звуковых сигналов. Кроме того, BIOS генерирует код текущего состояния загрузки (и, в случае сбоя, соответственно ошибки), который можно узнать при помощи комбинации светодиодов или семисегментных индикаторов (на некоторых материнских платах), а также на POST Card --- плате, которая вставляется в слот расширения на материнской плате (либо уже встроена в нее) и отображает код ошибки на своем индикаторе.
 
Сопоставить конкретный звуковой код, текстовое сообщение на мониторе или код POST с причиной сбоя во время загрузки компьютера можно по документации производителя BIOS, материнской платы или дополнительной платы контроллера устройства.
