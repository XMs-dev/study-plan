\section{Компьютер: что это?}\label{base:introduction:computer}
\subsection{Введение}\label{base:introduction:computer:introduction}
\emph{Компьютер} --- устройство или система, способная выполнять заданную, чётко определённую последовательность операций. Это чаще всего операции численных расчётов и манипулирования данными, однако сюда относятся и операции ввода-вывода. Описание последовательности операций называется \emph{программой}.
\emph{Электронная вычислительная машина}, или \emph{ЭВМ} --- комплекс технических средств, предназначенных для автоматической обработки информации в процессе решения вычислительных и информационных задач.

Название <<ЭВМ>>, принятое в русскоязычной научной литературе, является синонимом компьютера. В настоящее время оно почти вытеснено из бытового употребления и в основном используется инженерами цифровой электроники, как правовой термин в юридических документах, а также в историческом смысле --- для обозначения компьютерной техники 1940--1980-х годов и больших вычислительных устройств, в отличие от персональных.

Электронная вычислительная машина подразумевает использование электронных компонентов в качестве её функциональных узлов, однако компьютер может быть устроен и на других принципах --- он может быть механическим, биологическим, оптическим, квантовым и~т.\,п., работая за счёт перемещения механических частей, движения электронов, фотонов или эффектов других физических явлений. Кроме того, по типу функционирования вычислительная машина может быть цифровой (ЦВМ) и аналоговой (АВМ).

\subsection{Классификация}\label{base:introduction:computer:classification}
Компьютеры --- устройства с весьма широкой областью применения, поэтому, чтоб хоть как-то ориентироваться во всём этом многообразии, приведём несколько возможных классификаций.
По классу выполняемых задач:
\begin{itemize}
 \item Универсальные;
 \item Специализированные.
\end{itemize}
По виду вычислительного процесса:
\begin{itemize}
 \item Аналоговые вычислительные машины (АВМ);
 \item Гибридные вычислительные системы (машины) (ГВМ, ГВС);
 \item Цифровые вычислительные машины.
\end{itemize}
По виду рабочей среды:
\begin{itemize}
 \item Квантовый компьютер;
 \item Механический компьютер;
 \item Пневматический компьютер;
 \item Гидравлический компьютер;
 \item Оптический компьютер;
 \item Электронный компьютер;
 \item Биологический компьютер.
\end{itemize}
По назначению:
\begin{itemize}
 \item Сервер;
 \item Рабочая станция;
 \item Персональный компьютер.
\end{itemize}

\subsubsection{Сервер}\label{base:introduction:computer:classification:server}

\subsubsection{Рабочая станция}\label{base:introduction:computer:classification:workstation}

\subsubsection{Персональный компьютер}\label{base:introduction:computer:classification:pc}
Персональный компьютер же можно разделить на следующие подкатегории:
\begin{description}
 \item[Настольный компьютер (десктоп)] (от англ. \emph{desktop} --- на рабочем столе) --- наиболее известная форма. Используется почти для всего: поиск информации, моделирование, рисование, создание музыки, игры, общение, просмотр видео, программирование, редактирование документов и многое другое.
 \item[Ноутбук (лэптоп)] (от англ. \emph{notebook} --- блокнот, блокнотный ПК, и \emph{laptop} --- на коленях; это более широкий термин) --- портативный персональный компьютер, в корпусе которого объединены типичные компоненты ПК, включая дисплей, клавиатуру и устройство указания (обычно сенсорная панель), а также аккумуляторные батареи.
 
  Основное преимущество ноутбуков перед десктопами можно охарактеризовать одним словом --- мобильность. Оно включает в себя:
  \begin{inparaenum}
   \item малый вес и габариты,
   \item отсутствие потребности во внешних устройствах,
   \item возможность автономной работы,
   \item возможность подключения к беспроводным сетям.
  \end{inparaenum}
  Также есть у них и недостатки. Начнём со специфичных для ноутбуков:
  \begin{enumerate}
   \item Высокое соотношение цена/производительность;
   \item Низкая максимальная производительность;
   \item Ограниченность модернизации;
   \item Проблемы совместимости с различными операционными системами.
  \end{enumerate}   
  Теперь перейдём к тем, которые свойственны всем мобильным уст\-ройствам:
  \begin{enumerate}
   \item Низкое качество встроенных компонентов;
   \item Малоэффективная система охлаждения;
   \item Повышенная вероятность поломки;
   \item Сложность ремонта.
  \end{enumerate}
  
  Существует 2 основные системы классификации ноутбуков, которые дополняют друг друга.
  \begin{enumerate}
   \item Классификация на основе размера диагонали дисплея:
         \begin{itemize}
          \item 17 дюймов и более --- <<\emph{замена настольного ПК}>> (англ. \emph{Desktop replacement}). Также можно встретить название \emph{дескноут} (\emph{desktop}$+$\emph{notebook});
          \item 14--16 дюймов --- массовые ноутбуки (специального названия для данной категории ноутбуков не предусмотрено);
          \item 11--13,3 дюйма --- субноутбуки;
          \item 9--11 дюймов --- ультрапортативные ноутбуки;
          \item 7--12,1 дюйма (не имеющие DVD-привода) --- нетбуки;
          \item 5--10 дюймов --- смартбуки.
         \end{itemize}
   \item Классификация на основе назначения ноутбука и технических характеристик устройства:
         \begin{itemize}
          \item Бюджетные ноутбуки;
          \item Ноутбуки среднего класса;
          \item Бизнес-ноутбуки;
          \item Мультимедийные ноутбуки;
          \item Игровые ноутбуки;
          \item Мобильная рабочая станция;
          \item Имиджевые ноутбуки;
          \item Защищённые ноутбуки;
          \item Ноутбуки с сенсорным дисплеем.
         \end{itemize}
  \end{enumerate}
 \item[Планшетный компьютер] --- собирательное понятие, включающее в себя различные типы устройств с сенсорным экраном. Планшетным компьютером можно управлять прикосновениями руки или стилуса. Клавиатура и мышь доступны не всегда.
  К планшетным компьютерам могут относиться следующие устройства:
  \begin{itemize}
   \item Планшетный персональный компьютер:
         \begin{itemize}
          \item Планшетный нетбук;
          \item Тонкий ПК;
          \item Ультрамобильный ПК;
         \end{itemize}
   \item Мобильное интернет-устройство;
   \item Интернет-планшет;
   \item Электронная книга;
  \end{itemize}
 \item[Игровая приставка] (также \emph{игровая консоль}) --- специализированное электронное устройство, разработанное и созданное для видеоигр. Наиболее часто используемым устройством вывода является телевизор или, реже, компьютерный монитор --- поэтому такие устройства и называют приставками, так как они приставляются к независимому устройству отображения.
 Портативные (карманные) игровые системы имеют собственное встроенное устройство отображения (ни к чему не приставляются), поэтому называть их игровыми приставками несколько некорректно;
 \item[Карманный компьютер (КПК)] (англ. \emph{PDA, Personal Digital Assis\-tant}, также \emph{Handheld computer}) --- портативное устройство, служащее мобильным органайзером и записной книжкой;
 \item[Коммуникатор] --- карманный ПК со встроенным модулем мобильной связи. Коммуникаторы обладают рядом недостатков по сравнению с обычными КПК, основной из которых --- меньшее время автономной работы;
 \item[Смартфон] --- мобильный телефон, обладающий некоторыми или всеми функциями КПК;
 \item[Носимый компьютер] --- компьютер, который можно носить с собой на теле (что-то среднее между наручными часами и ноутбуком). На данный момент нет чёткой спецификации и стандартов для данного устройства. Наиболее предполагаемая научная область применения --- медицинские работники и военные.
\end{description}
