\section{Компьютер: что это?}\label{base:introduction:computer}
\subsection{Введение}\label{base:introduction:computer:introduction}
\emph{Компьютер} --- устройство или система, способная выполнять заданную, чётко определённую последовательность операций. Это чаще всего операции численных расчётов и манипулирования данными, однако сюда относятся и операции ввода-вывода. Описание последовательности операций называется программой.
\emph{Электронная вычислительная машина}, или \emph{ЭВМ} --- комплекс технических средств, предназначенных для автоматической обработки информации в процессе решения вычислительных и информационных задач.

Название <<ЭВМ>>, принятое в русскоязычной научной литературе, является синонимом компьютера. В настоящее время оно почти вытеснено из бытового употребления и в основном используется инженерами цифровой электроники, как правовой термин в юридических документах, а также в историческом смысле --- для обозначения компьютерной техники 1940--1980-х годов и больших вычислительных устройств, в отличие от персональных.

Электронная вычислительная машина подразумевает использование электронных компонентов в качестве её функциональных узлов, однако компьютер может быть устроен и на других принципах --- он может быть механическим, биологическим, оптическим, квантовым и~т.\,п., работая за счёт перемещения механических частей, движения электронов, фотонов или эффектов других физических явлений. Кроме того, по типу функционирования вычислительная машина может быть цифровой (ЦВМ) и аналоговой (АВМ).

\subsection{Классификация}\label{base:introduction:computer:classification}
Компьютеры --- устройства с весьма широкой областью применения, поэтому, чтоб хоть как-то ориентироваться во всём этом многообразии, приведём несколько возможных классификаций.
По классу выполняемых задач:
\begin{itemize}
 \item Универсальные;
 \item Специализированные.
\end{itemize}
По виду вычислительного процесса:
\begin{itemize}
 \item Аналоговые вычислительные машины (АВМ);
 \item Гибридные вычислительные системы (машины) (ГВМ, ГВС);
 \item Цифровые вычислительные машины.
\end{itemize}
По виду рабочей среды:
\begin{itemize}
 \item Квантовый компьютер;
 \item Механический компьютер;
 \item Пневматический компьютер;
 \item Гидравлический компьютер;
 \item Оптический компьютер;
 \item Электронный компьютер;
 \item Биологический компьютер.
\end{itemize}
По назначению:
\begin{itemize}
 \item Сервер;
 \item Рабочая станция;
 \item Персональный компьютер.
\end{itemize}

\subsubsection{Персональный компьютер}\label{base:introduction:computer:classification:pc}
Персональный компьютер же можно разделить на следующие подкатегории:
\begin{description}
 \item[Настольный компьютер (десктоп)] (от англ. \emph{desktop} --- на рабочем столе) --- наиболее известная форма. Используется почти для всего: поиск информации, моделирование, рисование, создание музыки, игры, общение, просмотр видео, программирование, редактирование документов и многое другое.
 \item[Ноутбук (лэптоп)] (от англ. \emph{notebook} --- блокнот, блокнотный ПК, и \emph{laptop} --- на коленях; это более широкий термин) --- портативный персональный компьютер, в корпусе которого объединены типичные компоненты ПК, включая дисплей, клавиатуру и устройство указания (обычно сенсорная панель), а также аккумуляторные батареи. 
\end{description}
