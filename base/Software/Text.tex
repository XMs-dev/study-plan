\section{Работа с текстом}\label{base:software:text}
\textbf{Emacs} --- один из мощнейших текстовых редакторов, обладающий полной свободой настройки и автоматизации. Обладает встроенным интерпритатором функционального языка Emacs Lisp на котором написан сам редактор и его расширения. Существуют две версии: GNU Emacs и XEmacs, однако сейчас различия между ними стираются. Также Emacs, как часть проекта GNU, обладет прекрасным переведённым руководсвом и базовыми уроками.

Рассмотрим обозначения. Если написано \Ctrl$-$\keystroke{S}, это означает, что надо с зажатой клавишей \Ctrl нажать клавишу \keystroke{S}, а если написано \Ctrl$-$\keystroke{X} \keystroke{1}, это означает, что с зажатой клавишей \Ctrl надо нажать \keystroke{X}, затем отпустить клавишb и нажать клавишу \keystroke{1}. В руководствах и справках принято обозначение \Ctrl --- \textbf{C} , \Alt - \textbf{M}. Буфер --- это что-то вроде вкладок в других редакторах. Редактор не удаляет текст, а перемещает его в специальный буфер, откуда его можно будет опять вставить - это называется <<убить>> текст.

\begin{itemize}
  \item \Ctrl$+$\keystroke{X} \Ctrl$+$\keystroke{F}  --- открыть файл;
  \item \Ctrl$+$\keystroke{X} \Ctrl$+$\keystroke{S}  --- сохранить файл;
  \item \Ctrl$+$\keystroke{X} \Ctrl$+$\keystroke{C}  --- выход;
  \item \Ctrl$+$\keystroke{X} \keystroke{S}          --- поиск;
  \item \Ctrl$+$\keystroke{X} \keystroke{2}  --- разделить текущее окно по вертикали;
  \item \Ctrl$+$\keystroke{X} \keystroke{3}  --- разделить текущее окно по горизонтали;
  \item \Ctrl$+$\keystroke{X} \keystroke{O}  --- перейти в соседнее окно emacs;
  \item \Ctrl$+$\keystroke{X} \keystroke{B}  --- указать буфер и перети в него;
  \item \Ctrl$+$\keystroke{X} \Ctrl$+$\keystroke{B}  --- список буферов;
  \item \Ctrl$+$\keystroke{X} \keystroke{U}  --- отмена;
  \item \Ctrl$+$\keystroke{K}  --- убить строку;
  \item \Ctrl$+$\keystroke{W}  --- убить выделенный фрагмент текста;
  \item \Ctrl$+$\keystroke{Y}  --- вставить последний фрагмент убитого текста;
  \item \Alt$+$\keystroke{Y}   --- перелистывать после вставки назад буфер убитого текста;
  \item \Ctrl$+$\keystroke{V} или \PgDown  --- листать вниз;
  \item \Alt$+$\keystroke{V}  или \PgUp    --- листать вверх;
  \item \Ctrl$+$\keystroke{A} или \Home  --- в начало строки;
  \item \Ctrl$+$\keystroke{E} или \End   --- в конец строки;
  \item \Alt$+$\keystroke{A}  --- в начало абзаца;
  \item \Alt$+$\keystroke{E}  --- в конец абзаца;
  \item \Ctrl$+$\LArrow  --- перейти на слово влево;
  \item \Ctrl$+$\RArrow  --- перейти на слово вправо;
  \item \Ctrl$+$\BSpace  --- удалить слово слева;
  \item \Ctrl$+$\Del     --- удалить слово справа;
  \item \Ctrl$+$\keystroke{H} \keystroke{I}  --- открыть info-страницы руководства;
\end{itemize}

Говорить о Emacs можно очень долго, советую пройти базовый урок, а также если вы захотите изучить какой-то редактор , необходимо запретить себе пользоваться остальными редакторами , хотя бы на время ,чтобы привыкнуть к управлению. Комбинации клавиш Emacs по перемещению и удалению текста поддерживаются многими программами, например Mozilla Firefox. Если вам понадобится среда для программирования на C  или Python с возможностью автодополнения и прочим прелестями IDE , есть неплохая уже готовая сборка \href{http://gabrielelanaro.github.com/emacs-for-python/}{emacs-for-python} , все настройки программы по умолчанию расположены в файле ~/.emacs и в каталоге ~/.emacs.d/